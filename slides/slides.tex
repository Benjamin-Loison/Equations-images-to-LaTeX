\documentclass{beamer}

\usepackage[french]{babel}
\usepackage[T1]{fontenc}
\usepackage[utf8]{inputenc}
\usepackage{moreverb}
\usepackage{enumitem}

\usetheme{Warsaw}

\newcommand{\NN}{\mathbb{N}}
\newcommand{\ttt}[1]{\texttt{#1}}

\setbeamertemplate{headline}{}

\makeatletter
\setbeamertemplate{footline}
{
  	\leavevmode%
  	\hbox{%
  	\begin{beamercolorbox}[wd=.30\paperwidth,ht=2.25ex,dp=1ex,center]{palette primary}%
    	\usebeamerfont{title in head/foot} \inserttitle
  	\end{beamercolorbox}%
  	
  	\begin{beamercolorbox}[wd=.50\paperwidth,ht=2.25ex,dp=1ex,center]{palette secondary}%
    	\usebeamerfont{title in head/foot} \insertauthor
  	\end{beamercolorbox}%
  
  	\begin{beamercolorbox}[wd=.15\paperwidth,ht=2.25ex,dp=1ex,center]{palette tertiary}%
    	\usebeamerfont{date in head/foot} \insertshortdate
  	\end{beamercolorbox}%
  
  	\begin{beamercolorbox}[wd=.05\paperwidth,ht=2.25ex,dp=1ex,center]{palette quaternary}%
    	\usebeamerfont{author in head/foot} \insertframenumber \hspace*{.1ex}
  	\end{beamercolorbox}}%
  	\vskip0pt%
}
\makeatother

\title{Converting equations to \LaTeX}
\author{Robin Sobczyk, Benjamin Loison, Bastien Lhopitallier}
\date{4 février 2022}
\institute{\href{https://github.com/Benjamin-Loison/Equations-images-to-LaTeX}{https://github.com/Benjamin-Loison/Equations-images-to-LaTeX}}

\AtBeginSection[]
{
\begin{frame}
\frametitle{Table of contents}
\tableofcontents[currentsection, hideothersubsections]
\end{frame}
}



\begin{document}



\begin{frame}
\titlepage
\end{frame}



%%%%%%%%%%%%%%%%%%%%%%%%%%%%%%%%%%
\section{Problem}
%%%%%%%%%%%%%%%%%%%%%%%%%%%%%%%%%%

\begin{frame}

    \frametitle{\LaTeX\;images conversion}
    \framesubtitle{\hfill Problem}
    
\begin{center}
    \[ \hat{f}(x) = \int^{1}_{0} \left( \dfrac{\partial f}{\partial x} f^{'}(x) \right) dx \]
    \[ \downarrow \]
    \texttt{\textbackslash hat \{ f \} ( x ) = \textbackslash int \textasciicircum~\{ 1 \} \_ \{ 0 \} \textbackslash left( \textbackslash dfrac \{ \textbackslash partial f \} \{ \textbackslash partial x \} f \textasciicircum~\{ ' \} ( x ) \textbackslash right) d x}
\end{center}

\end{frame}

%%%%%%%%%%%%%%%%%%%%%%%%%%%%%%%%%%
\section{Journey}
%%%%%%%%%%%%%%%%%%%%%%%%%%%%%%%%%%

\begin{frame}

    \frametitle{Handwritten dataset}
    \framesubtitle{\hfill CROHME-2014}
    
\begin{center}
\begin{itemize}[label=$\bullet$]
  \item INKML to RGB
  \item \includegraphics[width=0.3\paperwidth]{Images/out.png}
  \pause
  \item Trace: 404 62, ..., 404 62, 402 62, ...
  
  \pause
  \item Remove legends
  \pause
  \item Crop images to black to reduce pixels number
  \pause
  \item Flip images because upside down
  \pause
  \item Resize nearest neighbor
  \pause
  \item + statistics to get max rectangle taken by black pixels
\end{itemize}
\end{center}

\end{frame}

\begin{frame}

    \frametitle{Handwritten dataset}
    \framesubtitle{\hfill CROHME-2014}
    
\begin{center}
\begin{itemize}[label=$\bullet$]
  \item INKML to RGB
  \item \includegraphics[width=0.3\paperwidth]{Images/out.png}
  \item Final result
  \item \includegraphics[width=0.3\paperwidth]{Images/worked.png}
  \item Label: \$x\_k xx\_k + y\_k yx\_k\$
  \item Finally abandonned because don't know how to normalize images
  % https://arxiv.org/pdf/1712.03991.pdf GRU-based Encoder-Decoder Approach with Attention for Online Handwritten Mathematical Expression Recognition
\end{itemize}
\end{center}

\end{frame}

\begin{frame}

    \frametitle{Handwritten dataset}
    \framesubtitle{\hfill CROHME-2014}
    
\includegraphics[width=0.3\paperwidth]{Images/20_em_42.png}
\includegraphics[width=0.3\paperwidth]{Images/26_em_99.png}
\includegraphics[width=0.3\paperwidth]{Images/18_em_7.png}
\includegraphics[width=0.3\paperwidth]{Images/32_em_220a.png}

\end{frame}

\begin{frame}

    \frametitle{Compiled \LaTeX~images dataset}
    \framesubtitle{\hfill im2latex-100k}

Need some preprocess to transform:
\newline
\includegraphics[width=0.3\paperwidth]{Images/transformations.png}
\begin{itemize}[label=$\bullet$]
    \item
        \texttt{ds \textasciicircum\{2\} = (1 - \{qcos{\textbackslash}theta{\textbackslash}over r\})}
    \item
        ~['d', 's', '\textasciicircum', '\{', '2', '\}', '=', '(', '1', '-', '\{', '{\textbackslash}frac', '\{', 'q', 'c', 'o', 's', '{\textbackslash}theta', '\}', '\{', 'r', '\}', '\}', ')']
    \item
        ~[0, 11, 31, 9, 5, 32, 8, 28, 23, 33, 6, 5, 18, 5, 34, 35, 36, 31, 37, 8, 5, 38, 8, 8, 24, 1]
    % numbers single digits not treated as separeted tokens
\end{itemize}

\end{frame}

\begin{frame}

    \frametitle{Compiled \LaTeX~images dataset}
    \framesubtitle{\hfill im2latex-100k}

Need some preprocess to transform:
\newline
\begin{itemize}[label=$\bullet$]
  \item Removed 5 very unexpected images to reduce black rectangle area used to reduce pixels number
  \item Likewise total pixels number from 64,000,000,000 to 11,764,000,000, it is a 5.44 factor
  \item Furthermore these 5 images weren't in train, nor test, nor validation set (10 \% of images are in this case)
  \item Pad images to normalize, using Python but slow so used C++ with multithreading
\end{itemize}

\end{frame}

\begin{frame}

    \frametitle{\LaTeX~images generated dataset}
    \framesubtitle{\hfill im2latex-100k}
    
\includegraphics[width=0.2\paperwidth]{Images/big2.png}
\includegraphics[width=0.2\paperwidth]{Images/authors.png}
\includegraphics[width=0.2\paperwidth]{Images/12.png}
\includegraphics[width=0.2\paperwidth]{Images/15.png}
White image and 19x: M a t h T y p e ! Z Z h x 4 7 ! c a a a d a G c b i a H W n W d b a W c b i G H R a q e a O G a g 2 Z a b a a a b u q a b e G a c a a a b i

\end{frame}

% dataloader: "(set_size, sentence_length, vocabulary_size)"

%%%%%%%%%%%%%%%%%%%%%%%%%%%%%%%%
\section{Architecture of the network}
%%%%%%%%%%%%%%%%%%%%%%%%%%%%%%%%

\begin{frame}
    \frametitle{Conduct of the model}
    \framesubtitle{\hfill Architecture of the network}

\hspace*{-0.7cm}\includegraphics[width=0.95\paperwidth]{Images/network.png}

\vspace{1cm}
\footnotesize

\begin{tabular}{l|l|l}
    green : data & yellow : convolution & orange : pooling\\
    blue : batch normalization & purple : embedding & light grey : RNN\\
    $\oplus$ : element-wise addition & dark grey : Gumbel softmax & red : LSTM
\end{tabular}

\end{frame}

% hfactory test: slow, don't converge, sin...

%%%%%%%%%%%%%%%%%%%%
\section{Results and limitations}
%%%%%%%%%%%%%%%%%%%%

\begin{frame}
    \frametitle{Results}
    \framesubtitle{\hfill Results and limitations}
    
    After hours of studies and understanding\pause, hours of implementation\pause\\
    We achieved to get\pause
    
    \Huge{No result, it failed miserably}
    \centering
    \includegraphics[height=4cm]{Images/failed.jpg}
    
\end{frame}

\begin{frame}
    \frametitle{Flaws in the dataset}
    \framesubtitle{\hfill Results and limitations}
    
    \begin{block}{Examples}
    \begin{itemize}[label=$\bullet$]
        \item pictures mostly composed of white pixels (even some that are completely blank)
        \item huge gaps between parts of the formulas
    \end{itemize}
    \end{block}
    
    \begin{exampleblock}{}
    \begin{center}
        \includegraphics[scale=0.15]{Images/empty1.png} \quad \includegraphics[scale=0.15]{Images/empty2.png}
    \end{center}
    \end{exampleblock}
    
\end{frame}

\begin{frame}
    \frametitle{Incoherences in the article}
    \framesubtitle{\hfill Results and limitations}
    
    \begin{center}
        \includegraphics[scale=0.15]{Images/encoder_decoder.png}
        \includegraphics[scale=0.15]{Images/blstm.png}
    \end{center}
    
\end{frame}

%%%%%%%%%%
% Source %
%%%%%%%%%%



\begin{frame}

\frametitle{Sources}

\footnotesize

\begin{thebibliography}{1}
\setbeamertemplate{bibliography item}[text]

\bibitem{Paper1}
    \href{https://arxiv.org/ftp/arxiv/papers/1908/1908.11415.pdf}{Zelun Wang \& Jyh-Charn Liu,
    \emph{Translating Math Formula Images to LaTeX Sequences Using Deep
    Neural Networks with Sequence-level Training}}
        
\bibitem{Paper2}
    \href{https://arxiv.org/pdf/1609.04938.pdf}{Deng Y., Kanervisto A., Ling J. \& Rush A. M.,
    \emph{Image-to-Markup Generation with Coarse-to-Fine Attention}}
    
\bibitem{Paper3}
    \href{https://arxiv.org/pdf/1712.03991.pdf}{Zhang J., Du J. \& Dai L.,
    \emph{A GRU-based Encoder-Decoder Approach with Attention for Online Handwritten Mathematical
    Expression Recognition}}
    
\bibitem{Paper4}
    \href{http://cs231n.stanford.edu/reports/2017/pdfs/815.pdf}{Genthial G. \& Sauvestre R.,
    \emph{Image to Latex}}
    
\bibitem{im2markup}
    \href{https://github.com/harvardnlp/im2markup}{\texttt{https://github.com/harvardnlp/im2markup}}

\bibitem{dataset1}
    \href{http://tc11.cvc.uab.es/datasets/CROHME-2014_2}{\texttt{http://tc11.cvc.uab.es/datasets/CROHME-2014\_2}}
    
\bibitem{dataset2}
    \href{http://lstm.seas.harvard.edu/latex/data/}{\texttt{http://lstm.seas.harvard.edu/latex/data/}}

\end{thebibliography}

\end{frame}



\end{document}
